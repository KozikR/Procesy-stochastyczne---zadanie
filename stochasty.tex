\documentclass[10pt,a4paper]{article}
\usepackage[utf8]{inputenc}
\usepackage[T1]{fontenc}
\usepackage{amsmath}
\usepackage{amsfonts}
\usepackage{amssymb}
\usepackage[polish]{babel}
\usepackage[cm]{fullpage}
\usepackage{tabularx}
\usepackage{array}
\usepackage{booktabs}
\usepackage{forloop}
\usepackage{pgffor}
\usepackage{fancyhdr}
\usepackage{multirow}
\usepackage[pdftex]{graphicx}

\author{Rafał Kozik}
\title{Procesy stochastyczne}

\begin{document}
\section*{Procesy stochastyczne}
\subsection*{Rafał Kozik gr. 2 Automatyka i Robotyka EAIiIB}
	\subsection*{Zadanie 5.}
	Dane jest równanie rekurencyjne $\textbf{x}_{k+1} = A \textbf{x}_k +B \textbf{u}_k$. Wiadomo, że $E[\textbf{u}_k] = \overline{u} $, $R_u(k_1, k_2)= \sigma^2 I \delta (k_1-k_2)$, 
	$A = \left( \begin{smallmatrix} 1&0 \\ 0&1 \end{smallmatrix} \right)$, $B = \left( \begin{smallmatrix} 0&0 \\ 1&0 \end{smallmatrix} \right)$. Wyznacz i opisz własności $\textbf{x}_k$.

\subsection*{Rozwiązanie równania rekurencyjnego}
Rozpisując wektor $\textbf{x}_k$ jako:
\begin{equation}
	\textbf{x}_k = \left( \begin{smallmatrix}
	 x_k \\ y_k
	\end{smallmatrix} \right)
\end{equation}	  
można zapisać równanie rekurencyjne:
\begin{equation}
	\left(
	\begin{matrix}
		x_{k+1}\\
		y_{k+1}		
	\end{matrix}
	\right) = 
	\left(
	\begin{matrix}
		x_{k}\\
		y_{k}+u_k		
	\end{matrix}
	\right)
\end{equation}
Rozwiązaniem tego równania jest:
\begin{equation}
	\left(
	\begin{matrix}
		x_k \\ y_k
	\end{matrix}
	\right) 
	 =
	\left(
	\begin{matrix}
		x_0 \\ y_0 + \sum\limits_{i=0}^{k-1}u_i
	\end{matrix}
	\right) 
\end{equation}
gdzie $x_0$ i $y_0$ są warunkami początkowymi.
\subsection*{Wartość oczekiwana}
\begin{equation}
E[\textbf{x}_k]=E\left[ \left(	\begin{matrix}
		x_0 \\ y_0 + \sum\limits_{i=0}^{k-1}u_i
	\end{matrix} \right) \right]
	=
	\left(	\begin{matrix}
		E[ x_0 ] \\ E[ y_0 + \sum \limits_{i=0}^{k-1}u_i ]
	\end{matrix} \right) 
	=
	\left(	\begin{matrix}
		x_0 \\ y_0 + \sum \limits_{i=0}^{k-1} E[ u_i ]
	\end{matrix} \right) 
		=
	\left(	\begin{matrix}
		x_0 \\ y_0 + k \overline{u}
	\end{matrix} \right)
\end{equation}

\subsection*{Macierz }
\begin{equation}
	R_x(k_1, k_2) = E[ \textbf{x}_{k_1} \textbf{x}_{k_2} ^T ] =
	E \left( 
		\begin{matrix}
			x_{k_1} x_{k_2} & x_{k_1} y_{k_2} \\
			y_{k_1} x_{k_2} & y_{k_2} y_{k_1} 
		\end{matrix}			
	\right)
	=
	\left( 
		\begin{matrix}
			E(x_{0}^2) & E(x_0(y_{0} + \sum \limits_{i=0}^{k_2-1} u_i)) \\
			E(x_0(y_{0} + \sum \limits_{i=0}^{k_1-1} u_i) )& E( (y_{0} + \sum \limits_{i=0}^{k_1-1} u_i)(y_{0} + \sum \limits_{i=0}^{k_2-1} u_i) )
		\end{matrix}			
	\right) =
\end{equation}

\begin{equation}
	= \left( 
		\begin{matrix}
			x_{0}^2 & x_0(y_{0} + k_2 \overline{u} ) \\
			x_0(y_{0} + k_1 \overline{u}) & E( (y_{0} + \sum \limits_{i=0}^{k_1-1} u_i)(y_{0} + \sum \limits_{i=0}^{k_2-1} u_i) )
		\end{matrix}			
	\right)
\end{equation}
Obliczenie wartości prawego dolnego elementu macierzy:
\begin{equation}
	E \left( \left( y_{0} + \sum \limits_{i=0}^{k_1-1} u_i \right) \left( y_{0} + \sum \limits_{i=0}^{k_2-1} u_i \right) \right) = 
	E\left(y_{0} ^2\right) + E\left(y_0\sum \limits_{i=0}^{k_1-1} u_i\right) + E\left(y_0\sum \limits_{i=0}^{k_2-1} u_i\right) + 
	E \left( \left( \sum \limits_{i=0}^{k_1-1} u_i \right) \left( \sum \limits_{i=0}^{k_2-1} u_i \right) \right) = 
	\end{equation}

\begin{equation}
     =y_0^2 + y_0(k_1+k_2) \overline{u}+ E \left( \left( \sum \limits_{i=0}^{k_1-1} u_i \right) \left( \sum \limits_{i=0}^{k_2-1} u_i \right) \right)
\end{equation}	

Z faktu, że prawy dolny element macierzy $R_u(k1, k2)$, czyli $ E[u_{k_1}u_{k_2}]]$ jest równy
$\sigma^2 \delta(k_1-k_2)$ wynika, że wartość oczekiwana z iloczynu dwóch zmiennych losowych jest różna od zera tylko wtedy, gdy $k_1 = k_2$, wieć w wyrażeniu $ \left( \sum \limits_{i=0}^{k_1-1} u_i \right) \left( \sum \limits_{i=0}^{k_2-1} u_i \right) $ będzye tylko $\min(k_1, k_2)$ iloczynówm których wartość oczekiwana jest różna od 0:
\begin{equation}
E\left( \left( \sum \limits_{i=0}^{k_1-1} u_i \right) \left( \sum \limits_{i=0}^{k_2-1} u_i \right) \right) = 
E\left( \sum \limits_{i=0}^{ \min(k_1, k_2) -1}  u_i^2 \right) = 	
\sum \limits_{i=0}^{ \min(k_1, k_2) -1} E\left(  u_i^2 \right) = 
\sum \limits_{i=0}^{ \min(k_1, k_2) -1} \sigma ^2 = \min( k_1, k_2 ) \sigma^2  
\end{equation}

Więc
\begin{equation}
R_x(k_1, k_2) =
\left( 
		\begin{matrix}
			x_{0}^2 & x_0(y_{0} + k_2 \overline{u} ) \\
			x_0(y_{0} + k_1 \overline{u}) & y_0^2 + y_0(k_1+k_2) \overline{u}+\min( k_1, k_2 ) \sigma^2
		\end{matrix}			
	\right)
\end{equation}

\subsection*{Macierz}
\begin{equation}
K_x(k_1, k_2) = R_x(k_1, k_2) - E \left( \textbf{x}_{k_1} \right) E \left( \textbf{x}_{k_2} \right) ^T
\end{equation}

\begin{equation}
E \left( \textbf{x}_{k_1} \right) E \left( \textbf{x}_{k_2} \right) ^T = 
\left(
\begin{matrix}
	x_0 \\ y_0 + k_1 \overline{u}
\end{matrix}
\right)
\left(
\begin{matrix}
	x_0 \\ y_0 + k_2 \overline{u}
\end{matrix}
\right)^T=
\left(
	\begin{matrix}
	x_0^2 & x_0 \left( y_0 + k_2 \overline{u} \right) \\
	x_0 \left( y_0+k_1 \overline{u} \right) &  y_0^2 + y_0 \overline{u} \left(k_1+k_2 \right) + k_1 k_2 \overline{u} 
	\end{matrix}
\right)
\end{equation}
wiec 
\begin{equation}
K_x(k_1, k_2) = 
\left( 
		\begin{matrix}
			x_{0}^2 & x_0(y_{0} + k_2 \overline{u} ) \\
			x_0(y_{0} + k_1 \overline{u}) & y_0^2 + y_0(k_1+k_2) \overline{u}+\min( k_1, k_2 ) \sigma^2
		\end{matrix}			
	\right)
-
\left(
	\begin{matrix}
	x_0^2 & x_0 \left( y_0 + k_2 \overline{u} \right) \\
	x_0 \left( y_0+k_1 \overline{u} \right) &  y_0^2 + y_0 \overline{u} \left(k_1+k_2 \right) + k_1 k_2 \overline{u} 
	\end{matrix}
\right)
\end{equation}
\begin{equation}
K_x(k_1, k_2) = 
\left(
	\begin{matrix}
	0 & 0 \\
	0 & \min \left( k_1, k_2 \right) \sigma ^2 - k_1 k_2 \overline{u} 
	\end{matrix}
\right)
\end{equation}
\subsection*{Wariancia}
\begin{equation}
V \left( \textbf{x}_k \right) = 
\left(
	\begin{matrix}
	0 \\ V \left( y_0 + \sum \limits_{i=0}^{k-1} u_i \right)
	\end{matrix}
\right) = 
\left(
	\begin{matrix}
	0 \\ V \left(\sum \limits_{i=0}^{k-1} u_i \right)
	\end{matrix}
\right)
\end{equation}

\begin{equation}
V \left(\sum \limits_{i=0}^{k-1} u_i \right) = 
E \left( \left( \sum \limits_{i=0}^{k-1} u_i \right)^2 \right) - E \left( \sum \limits_{i=0}^{k-1} u_i \right)^2 =
E \left( \sum \limits_{i=0}^{k-1} u _i ^2  \right) + E \left( \sum \limits_{i,j=0 \; i \neq j} ^{k-1} 2u_i u_j \right) - E^2 \left( \sum \limits_{i=0}^{k-1} u_i \right)
\end{equation}

Jak było wspomniane wcześniej korzystając ze znanej macierzy $R_u$ można obliczyć:

\begin{equation}
E \left( \sum \limits_{i=0}^{k-1} u_i^2  \right) = k \sigma ^2 \qquad E \left( \sum \limits_{i,j=0 \; i \neq j}^{k-1} 2u_i u_j \right) = 0 
\end{equation}
więc 
\begin{equation}
	V \left( \textbf{x}_k \right) = 
	\left(
	\begin{matrix}
	0 \\ 
	k\sigma^2 - k^2 \overline{u}^2	
	\end{matrix}
\right)
\end{equation}
\end{document}